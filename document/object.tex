
%%% Local Variables: 
%%% mode: latex
%%% TeX-master: "russian"
%%% End: 

\section{Gtk编程特点}

\subsection{类型转换}

GTK+一般使用c语言来撰写,c语言不支持面向对象,但GTK使用一些方式,支持一点面向对象
概念。GTK以结构体(struct)模拟类(class);新建类用gtk\_xxx\_new(type)的方式;使 用函
数名区分毎组类的方法,与GtkWindow相关的方法,都以gtk\_window开 头,如要设
置window的标题,使用:

\begin{shell}
\begin{verbatim}
gtk_window_set_title(GTK_WINDOW(window), "awareness");
\end{verbatim}
\end{shell}

gtk\_window\_set\_title第一个参数接受GtkWindow指针,通过这种方式,一组公用方法就
专 属GtkWindow类使用了。GTK的类中只有数据,没有方法,调用相关方法要传递类指针;由
于没有多态概念,传入 的指针必须转换为指定类型,如GTK\_WINDOW,把指针转换
为GtkWindow类型,GTK\_WINDOW是一个宏,用来进行指针类型转换,定义如
下 gtkwindow.h(usr/include/gtk-2.0/gtk)中:

\begin{shell}
\begin{verbatim}
#define GTK_WINDOW(obj)                                                       
        (G_TYPE_CHECK_INSTANCE_CAST((obj), GTK_TYPE_WINDOW, GtkWindow))
\end{verbatim}
\end{shell}

G\_TYPE\_CHECK\_INSTANCE\_CAST宏定义在GLib的 gtype.h(usr/include/glib-2.0/gobject)中:

\begin{shell}
\begin{verbatim}
#define G_TYPE_CHECK_INSTANCE_CAST(instance, g_type, c_type)
    (_G_TYPE_CIC ((instance), (g_type), c_type)) 
\end{verbatim}
\end{shell}

G\_TYPE\_CHECK\_INSTANCE\_CAST宏会检查instance是否为g\_type的一个实例,如果不是 的话就
发出警示讯息,若是的话就将指标转型为c\_type型态。GTK实际上使用结构链接(link)的方
式,GTK中的类继承关系如下:

\begin{shell}
\begin{verbatim}
GObject                                                                       
 +--GInitiallyUnowned                                                         
     +-- GtkObject                                                            
           +-- GtkWidget                                                      
                 +-- GtkContainer                                             
                       +-- GtkBin                                             
                             +-- GtkWindow  
\end{verbatim}
\end{shell}

G\_TYPE\_CHECK\_INSTANCE\_CAST就是根据这继承关系作转换的,在GTK定义中,毎个高 层次的
类都会包含低层次类,如:

\begin{shell}
\begin{verbatim}
struct _GtkWindow
{
  GtkBin bin;

  gchar *GSEAL (title);
  gchar *GSEAL (wmclass_name);
  gchar *GSEAL (wmclass_class);
  gchar *GSEAL (wm_role);

  GtkWidget *GSEAL (focus_widget);
  GtkWidget *GSEAL (default_widget);
  GtkWindow *GSEAL (transient_parent);
  GtkWindowGeometryInfo *GSEAL (geometry_info);
  GdkWindow *GSEAL (frame);
  GtkWindowGroup *GSEAL (group);

  guint16 GSEAL (configure_request_count);
  guint GSEAL (allow_shrink) : 1;
  guint GSEAL (allow_grow) : 1;
  guint GSEAL (configure_notify_received) : 1;
  ...
}

struct _GtkContainer                                                          
{                                                                             
  GtkWidget widget;                                                           

  GtkWidget *focus_child;                                                     

  guint border_width : 16;                                                    

  /*< private >*/                                                             
  guint need_resize : 1;                                                      
  guint resize_mode : 2;                                                      
  guint reallocate_redraws : 1;                                               
  guint has_focus_chain : 1;                                                  
}; 
\end{verbatim}
\end{shell}

GtkWindow的成员有一个GtkBin,GtkContainer的成员中有一个GtkWidget。

\subsection{事件处理模式}

\subsubsection{事件和信号}

所有图形环境的事件都是从系统(Gnome)传递来的,Gtk当然也是这样,事件通常由用户输
入产生,如按键、鼠标,或控件焦点改变。当你按下鼠标键,系统(Gnome)就产生事件,并
放入事件队列中,系统根据鼠标点击的坐标确定是哪个GtkWidget对象产生的事件,如按下
的是button按钮,事件结构体包将含有坐标信息,产生事件的button指针。

事件分为event和signal,event来自GDK事件体系,GDK事件又来自于系统。信signal是二次
包装后添加到GTK中的,再次包装的原因是GDK事件不够健壮、灵活。本质上
说,event是Gdk/Xlib的概念,event是从系统传递来的信息流,event放入队列,在GTK主循
环中(main loop)处理。signal是GtkWidget的子类,当有鼠标按动时,并不产生信号,信
号只是记录与之相关的回调函数(callback),signal由GtkWidget对象发出,和GTK主循环
无关。总之,event产生于系统,用GDK传递到程序,signal由产生GtkWidget对象发出,比
如,用鼠标按下button按钮时,系统在button上产
生``button\_press\_event'',接着button发出``cilcked signal''。并不是所有事件都有
对应信号,如拖曳(drag\_event)就没有对应的signal;相应的,并不是所有signal都有对
应event ,GtkWidget对象能产生自己的signal。

\subsubsection{事件定义}

为了连结一个事件与Callback函数,一样是使用g\_signal\_connect(),不过处
理event的Callback函式与signal的Callback函式在宣告时有些不同,以下是Callback函数定
义:

\begin{shell}
\begin{verbatim}
// 处理event回调函数
gboolean callback_func(
          GtkWidget *widget, GdkEvent *event, gpointer callback_data);
// 处理signal回调函数
void callback_func(
          GtkWidget *widget,  gpointer callback_data);
\end{verbatim}
\end{shell}

处理event的Callback函数多了一个GdkEvent*参数,而返回值的部份,可以控制事件是否进行
下一步传播,传回TRUE表示这个事件到止已获得处理,事件不用继续传播,将不会发出对应
的signal,传回FALSE表示事件继续传播,GtkWidget对象发出对应的signal。

event的处理函数会在signal的处理函数之前先处理,以按下按钮为例,基本上的
顺序为:
\begin{shell}
\begin{verbatim}
按钮按下 --> 发出 button_press_event --> 预设处理函数
                 --> 发出 clicked signal --> 预设处理函数
\end{verbatim}
\end{shell}

您可以设置事件的Callback函数,拦截button\_press\_event,当处理完传回TRUE时,就不
会发出clicked的signal,也就不会继续预设的signal处理函数,只有在传回FALSE时,才会
发出clicked的signal,则设置的signal处理函数才会被执行。
