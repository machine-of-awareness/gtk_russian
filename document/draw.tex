
%%% Local Variables: 
%%% mode: latex
%%% TeX-master: "russian"
%%% End: 

\section{Gtk绘图方式}

GtkWidget对象中,GtkDrawingArea是专门用于绘图的,不用设置它的GTK\_APP\_PAINTABLE
标记,它的默认expose\_event回调函数不绘制形状,不会覆盖你自己绘制的图形。

在Gtk中,可以在GtkWidget对象上绘图,但必须设置GTK\_APP\_PAINTABLE标记,否
则,当对象接收到对象默认的expose\_event事件,你绘的图会被覆盖,当对象发生
expose\_event事件时,事件传递过程是这样的:
\begin{itemize}
\item 自己定义的expose\_event回调函数运行,在函数里绘制需要的图形
\item 对象默认的expose\_event回调函数运行,如果对象的GTK\_APP\_PAINTABLE标记未
  置位,会在函数里绘制对象默记形状,你自己绘制的图形会被覆盖;如果对象
  的GTK\_APP\_PAINTABLE标记置位,函数不会在函数里绘制对象默记形状
\item 父类的expose\_event回调函数运行
\end{itemize}

绘图命令将图形绘制在GdkDrawable对象上,GdkDrawable是一个接口,支持绘制图
形,包括GdkWindow,GdkPixmap,GdkBitmap对象,GdkDrawable是GObject的子类。Gtk绘图
命令有:绘点、直线、弧线、多边形、文字、图片等。绘图的内容可以在GdkDrawable对象间
方便的复制,也可以在GdkImage、GdkPixbuf之间复制,这些使GdkDrawable成为功能强大的
对象。

所有的绘图方法操作GdkDrawable对象,并将GdkDrawable对象作为方法的第一个参数。由于
绘图中经常要指定绘图参数,如绘制的前景色、背景色、线宽等,GDK定义了GdkGC结构包含
这些信息,这样用一个GdkGC参数,就可以将各种参数传递给绘图方法。

\subsection{内存中绘图}

直接在GtkWidget上绘图是简单的,当expose\_event事件发生时就有麻烦了,你需要记住绘
制过的所有图形,然后在事件回调函数中重新绘制,这是很讨厌的。解决的办法是将图形绘
制在内存中,但并不显示,当需要显示到屏幕或有expose\_event事件发生时,从内存复制相应
的内容到GtkWidget对象中。我们直接操作的对象是GdkPixmap,是GdkDrawable的子
类,GdkPixmap是资源对象。

下面是生成GdkPixmap的方法,drawable参数是对应显示图形的GdkDrawable对象,一般
是GtkWidget对象的GdkWindow对象(widget$\rightarrow{}$window),显示到屏幕就是
把GdkPixmap中的内容复制到drawable中。depth表示绘图毎一点占位的大小,安全的方式是
置成-1,GdkPixmap会自动设制位的大小。

\begin{shell}
\begin{verbatim}
GdkPixmap* gdk_pixmap_new (GdkDrawable *drawable, // to determine depth of pixmap
                           gint width,            // width of pixmap in pixels
                           gint height,           // height of pixmap in pixels
                           gint depth);           // bits per pixel
\end{verbatim}
\end{shell}

当将GdkPixmap中的内容(图形)显示到屏幕上(GtkDrawingArea的GdkWindow),需要调
用gdk\_draw\_drawable方法,调用后,会自动将GdkPixmap内容复制到GdkWindow中。
xsrc,ysrc是源(GdkPixmap)的起始坐标,xdest,ydest是目标(GdkWindow)的起始地
址,width,height是复制区域大小,如果要复制整个区域,可以将width或height设成-1。

\begin{shell}
\begin{verbatim}
void gdk_draw_drawable (GdkDrawable *target_drawable,
                        GdkGC *gc,
                        GdkDrawable *src,
                        gint xsrc,
                        gint ysrc,
                        gint xdest,
                        gint ydest,
                        gint width,
                        gint height);
\end{verbatim}
\end{shell}

绘图的过程一般是这样的,先将图形绘制到GdkPixmap中,然后调
用gtk\_widget\_queue\_draw\_area方法,方法中指定需要重绘的区域,系统会将这个区域
置成invalidate状态,置invalidate状态会触发expose\_event,expose\_event回调函数参
数中有区域信息,调用gdk\_draw\_drawable方法,将GdkPixmap中指定区域复制
到GdkWindow的指定位置,图形就显示出来了。方法中的x,y是置成invalidate状态的区域起
始坐标,width,height是区域大小。

如果整个GdkDrawingArea被遮住又重新显示,系统会自动产生expose\_event事件,回调函数
参数有区域信息,就是GdkDrawingArea的全部区域。

\begin{shell}
\begin{verbatim}
void gtk_widget_queue_draw_area (GtkWidget *widget,
                                 gint x,
                                 gint y,
                                 gint width,
                                 gint height);
\end{verbatim}
\end{shell}
